% Sun Yihan's Academic Curriculum Vitae Latex Template
%
% (c) 2002 Matthew Boedicker <mboedick@mboedick.org> (original author) http://mboedick.org
% (c) 2003-2007 David J. Grant <davidgrant-at-gmail.com> http://www.davidgrant.ca
% (c) 2007-2014 Todd C. Miller <Todd.Miller@sudo.ws> http://www.sudo.ws/todd
% (c) 2021 Z.C. TIAN <ztian002@e.ntu.edu.sg> https://github.com/doem97
% This work is licensed under the Creative Commons Attribution-ShareAlike 3.0 Unported License. To view a copy of this license, visit http://creativecommons.org/licenses/by-sa/3.0/ or send a letter to Creative Commons, 171 Second Street, Suite 300, San Francisco, California, 94105, USA.

\documentclass[a4paper,11pt]{article}

%-----------------------------------------------------------
% Include packages
\usepackage[empty]{fullpage}
\usepackage[UTF8]{ctex}
\usepackage{color}
\usepackage{verbatimbox}
\usepackage{hyperref}
\definecolor{mygrey}{gray}{0.85}
\raggedbottom
\raggedright
\setlength{\tabcolsep}{0mm}
% \renewcommand{\arraystretch}{1.1}
\pagestyle{plain}

% Adjust margins to 12.5mm on all sides
\addtolength{\oddsidemargin}{-12mm}
\addtolength{\evensidemargin}{-12mm}
\addtolength{\textwidth}{25mm}
\addtolength{\topmargin}{-12mm}
\addtolength{\textheight}{25mm}

%-----------------------------------------------------------
% Custom commands
% \renewcommand{\labelitemii}{$\bullet$} % Change the style of symbol in front of the item
% \renewcommand{\labelitemiii}{$-$}
\newcommand{\lsitem}[1]{\item #1 \vspace{-4pt}}
\newcommand{\resheading}[1]{{ \vspace{2pt} \pdfbookmark{#1}{#1}
\large \colorbox{mygrey}{\begin{minipage}{\textwidth}{\textbf{#1 \vphantom{p\^{E}}}} \end{minipage}}
}}

\newcommand{\resschoolheading}[5]{
\addvbuffer[2pt 2pt]{
\begin{tabular*}{184mm}{l@{\extracolsep{\fill}}r}
        % \bf \textsl{#1} & #2 \\
        \textbf{#1} & \textbf{#2} \\
		\textbf{#3} & \textbf{#4} \\
		\textit{#5}
\end{tabular*}}
} % 教育经历的5个控制模块: {学校}{地点}{学院及成绩}{就读时间}{主修课程}
\newcommand{\resschoolheadingtwo}[4]{
\addvbuffer[2pt 2pt]{
\begin{tabular*}{184mm}{l@{\extracolsep{\fill}}r}
        % \bf \textsl{#1} & #2 \\
        \textbf{#1} & \textbf{#2} \\
		\textbf{#3} & \textbf{#4} \\
\end{tabular*}}
} % 教育经历的4个控制模块: {学校}{地点}{学院及成绩}{就读时间}


\newcommand{\resfourheading}[2]{
\addvbuffer[0pt 0pt]{
\begin{tabular*}{184mm}{l@{\extracolsep{\fill}}r}
        % \bf \textsl{#1} & #2 \\
        \textbf{#1} & \text{#2} \\
\end{tabular*}}
} % 实习经历的4个控制模块: {项目名称}{时间}

\newcommand{\resfiveheading}[4]{
\addvbuffer[0pt 0pt]{
\begin{tabular*}{184mm}{l@{\extracolsep{\fill}}r}
        % \bf \textsl{#1} & #2 \\
        \textbf{#1} & \text{#3} \\
		\textbf{#4} & \text{#2}
\end{tabular*}}
} % 实习经历的4个控制模块: {公司}{地点}{时间}{职位}

\newcommand{\resaward}[2]{
\addvbuffer[1pt 1pt]{
\begin{tabular*}{184mm}{l@{\extracolsep{\fill}}r}
        {#1} & \textit{#2}\\
\end{tabular*}}
} % 获奖模块:{获奖或证书}{时间}
%-----------------------------------------------------------


\begin{document}
% \addvbuffer[0pt 3pt]
{
\def\arraystretch{1.5}
\begin{tabular*}{185mm}{l@{\extracolsep{\fill}}r}
        \textbf{\LARGE 孙怡晗}  & 联系方式+65-8900-8490\\
          出生日期05-08-1998 &  邮箱:SUNY0045@e.ntu.edu.sg\\
        \ 民族:汉族 & 学历:硕士研究生\\
\end{tabular*}
}
\vspace{0.25mm}

%%%% 教育经历
\resheading{教育经历}
\resschoolheading{新加坡南洋理工(NTU)}{新加坡}{电子与电气学院通信工程专业(GPA: 4.0/5.0)}{2021.1-2022.1}{主修课程:机器学习,无线通信,光纤通信,分布式多媒体,射频信号}
\resschoolheading{四川大学(SCU)}{四川省成都市}{电子信息学院电子工程专业(GPA: 3.3/4.0)}{2016.9-2020.6}{主修课程:数字电路,模拟电路,数字信号处理,信号与系统,电磁场与微波}
\resschoolheadingtwo{新加坡国立大学(NUS)夏校访学}{新加坡}{计算机工程(GPA:A(excellent))}
{2019.6-2020.8}
\resschoolheading{台湾国立金门大学(NQU)}{台湾省}{电子工程学院电子工程系(GPA:4.88/5)}
{2019.1-2019.6}{主修课程:机器人入门,积体电路的绘制,智慧型计算,IOS程序设计}

%%%% 获奖及证书
\resheading{获奖及证书}
\resaward{全球大学生数学建模大赛全球二等奖}{2019}
\resaward{全国大学生创新创业优秀项目奖}{2018}
\resaward{模拟联合国优秀国家代表}{2017}
\resaward{中国红十字会优秀志愿者}{2017}
\resaward{校级单项奖学金}{2019}
\resaward{雅思IELTS学术类(A类)6.5}{2020}
\resaward{CET4及CET6证书}{2016/2018}


%%%% 实习经历
\resheading{实习经历}

\resfiveheading{Ultra-Wireless Pte Ltd}{新加坡}{2021.6-至今}{硬件算法实习生(RTLS\&UWB定位)}
\begin{itemize}
    \item 对RTLS定位模块的算法进行改进,利用IEEE802.15标准和串口协议解析脚本,提高定位精度
    \begin{itemize}
        \item 测试现有定位模块,进行数据解析,偏差补偿,漂移量控制
        \textit{: 测试不同高度,环境的RTLS定位模块,编写脚本解析log得到数据偏移量及标准差并进行矫正,提高测距准确度,利用双向飞行距离算法定位target模块,选用了集中度大于90\%的数据分析减少无关项的干扰提高分析效率。}
        \item 利用Verilog编写makefile文件实现模块,协助进行雷达定位的测试与串口文件分析。
        \end{itemize}
\end{itemize}

\resfiveheading{FroForest.Pte.Ltd}{北京}{2020.8-2021.1}{NLP实习生(OCR方向)}
\newpage
\begin{itemize}
    \item 基于NLP及分词技术的OCR算法改进
    \begin{itemize}
        \item 测试开源OCR在识别化学检验文档时准确度,\textit{:调用百度OCR开源api对公司实验文档进行初步识别,统计并分析错误概率较高的字符或数据类型出现的概率及原因。}
        \item 利用NLP和分词技术修正OCR识别结果\textit{:利用NLP和分词训练语料库,减少词组中错误概率;映射替换字符提高易错字及化学符号的识别率,修正算法后将识别率由75\%提升至89\%.}
    \end{itemize}
\end{itemize}


\resfiveheading{清华大学物理实验室\&普测时空}{北京}{2019.12-2020.4}{电子硬件实习生}
\begin{itemize}
    \item 利用FPGA及Xilinx电路板对锁相环进行测试验证,设计PCB电路板实现十倍频率电路。\vspace{-0.2cm}
    \begin{itemize}
        \item 利用Verilog编写testbench进行电路验证\textit{:编写teshbench产生激励,观测不同边界条件下的输出及相移产生原因。测试由于温漂及电路频率较高时产生的偏差和损耗。}
        \item 制作十倍频PCB电路\textit{:利用5倍频和2倍频元件实现高频电路的设计,利用多层结构电路板及布线方式减少电磁信号的干扰并完成打板。}
    \end{itemize}
\end{itemize}

\resfiveheading{长虹电器股份有限公司}{四川}{2019.8-2019.12}{实习软件工程师}
\begin{itemize}
    \item 通过python解析MySQL日志,上传至Hive及Hadoop管理。\vspace{-0.2cm}
    \begin{itemize}
        \item 配置、获取并解析MySQL Binlog文档并在Hive及Hadoop数据仓库进行管理\textit{:编写python脚本获取部门数据库Binlog文档并解析为可读模式,写入HDFS后导入Hive,在Hadoop进行数据管理和编辑。}
    \end{itemize}
\end{itemize}

%%%% 校内项目经历
\resheading{校内项目经历}
\resfourheading{智能健康水杯底座设计与开发}{2016.12-2018.4}
\begin{itemize}
    \lsitem{利用压力传感器记录饮水量,通过蓝牙传输到手机端,在APP显示并提示用户是否到达每日饮水量}
    \vspace{-0.2cm}
    \begin{itemize}
        \item 负责蓝牙模块的选型及测试以及水杯底座电路板的制作\textit{:由压力传感器得到数据,利用CC254系列蓝牙配置寄存器并接收串口数据,通过中断回调函数记录总量。}
        \item 制作PCB电路板\textit{:电路仿真实现功能后,根据电路图进行PCB布线及制作,完成电路打板。}
    \end{itemize}
\end{itemize}

\resfourheading{图像识别及垃圾分类APP开发}{2019.7~2019.8}
\begin{itemize}
    \lsitem{利用IBMcloud,Node.js编写分类器,在数据库进行模糊查询得到主体内容信息\textit{:爬取不同垃圾图片,调用IBMcloud已有API训练模型。在测试集返回图片主体内容关键字,在数据库进行模糊查询返回垃圾类别及降解时间信息。}}
\end{itemize}

%%%% 社会实践及活动
\resheading{社会实践及活动}
\begin{itemize}
% \setlength{\itemsep}{-0.2cm}
    \item 体育舞蹈协会外联部干事:2016年12月作为主持人参与干训会并主持舞蹈晚会。
    \item 电子信息学院团委创就部干事:2016年12月参与并组织学院迎新晚会。
    \item 2016年-2017年参与中国红十字会“爱心包裹”募捐并获得“优秀志愿者”称号
\end{itemize}

%%%% SKILLS
\resheading{自我评价}
\begin{itemize}
    \item 本人善于学习,有较好抗压能力和创新能力;乐于沟通学习。
\end{itemize}

\end{document}